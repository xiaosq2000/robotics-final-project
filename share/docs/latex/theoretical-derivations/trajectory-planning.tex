% Tufte style handout
\documentclass{tufte-handout}
% Book metadata
\title{轨迹规划}
\author{肖书奇}
% Uncomment this line if you prefer colored hyperlinks (e.g., for onscreen viewing)
\hypersetup{colorlinks}              
% Compatibility with xelatex engine
\usepackage{ifxetex}
\ifxetex
    \newcommand{\textls}[2][5]{%
        \begingroup\addfontfeatures{LetterSpace=#1}#2\endgroup
    }
    \renewcommand{\allcapsspacing}[1]{\textls[15]{#1}}
    \renewcommand{\smallcapsspacing}[1]{\textls[10]{#1}}
    \renewcommand{\allcaps}[1]{\textls[15]{\MakeTextUppercase{#1}}}
    \renewcommand{\smallcaps}[1]{\smallcapsspacing{\scshape\MakeTextLowercase{#1}}}
    \renewcommand{\textsc}[1]{\smallcapsspacing{\textsmallcaps{#1}}}
\fi
% 支持中文
\usepackage{ctex}
\usepackage{xeCJK}
\xeCJKsetup{PunctStyle=kaiming}  % 设置中文标点符号,开明式
% \xeCJKsetup{PunctStyle=quanjiao} % 设置中文标点符号,全角式
\setCJKmainfont{仓耳今楷05-6763}      % 设置正文罗马族的 CJK 字体,影响 \rmfamily 和 \textrm 的字体。
% \setCJKsansfont{}  % 设置正文无衬线族的 CJK 字体,影响 \sffamily 和 \textsf 的字体。
\setCJKmonofont{SimHei}          % 设置正文等宽族的 CJK 字体,影响 \ttfamily 和 \texttt 的字体。
% Supports AMS math
\usepackage{amsmath}
\usepackage{amssymb}
\usepackage{amsfonts}
% For nicely typeset tabular material
\usepackage{booktabs}
% For graphics / images
\usepackage{graphicx}
    \setkeys{Gin}{width=\linewidth,totalheight=\textheight,keepaspectratio} % Set default scale globally
    \graphicspath{{graphics/}}        % Set paths to search for images
\begin{document}
\maketitle
% \tableofcontents   % TOC
\section{操作空间、梯形速度}
\marginnote[13cm]
{
    \begin{enumerate}
        \item
              输入梯形速度规划六参数
              \begin{itemize}
                  \item 起点位姿
                  \item 终点位姿
                  \item 采样时间
                  \item 最大速度
                  \item 加速阶段加速度
                  \item 减速阶段加速度
              \end{itemize}
        \item
              求出加速、减速、匀速三阶段所需时间
        \item
              定义方向向量和采样点数
        \item
              确定加速阶段结束点、减速阶段起始点的序号
        \item
              代入运动学公式
        \item
              转为递推形式
    \end{enumerate}
}
\[
    \vec x_0,\ \vec x_1,\ t_s,\ v,\ a_{acc},\ a_{dec}
\]
\[
    t_{acc} := \frac{v}{a_{acc}},\ t_{dec} := \frac{v}{a_{dec}},\ t_v := \left.\left(\vert x\vert-\frac{v^2}{2a_{acc}}-\frac{v^2}{2a_{dec}}\right)\right/v
\]
\[
    \vec k = \frac{\vec x}{\vert x\vert},\ N:=\frac{t}{t_s}
\]
\[
    N_{acc} = N\frac{t_{acc}}{t},\ N_{dec} = N(1-\frac{t_{dec}}{t})
\]
\[
    \vec x(n) = \left\{\begin{aligned}
         & \dfrac{1}{2}\vec k a_{acc}(nt_s)^2                                                                                     &  & 0<n<N_{acc}       \\
         & \dfrac{1}{2}\vec k a_{acc}(N_{acc}t_s)^2 + \vec k v(nt_s-t_{acc})                                                      &  & N_{acc}<n<N_{dec} \\
         & \dfrac{1}{2}\vec k a_{acc}(N_{acc}t_s)^2 + \vec k v(N_{dec}t_s-t_{acc}) - \frac{1}{2}\vec k a_{dec}(nt_s-N_{dec}t_s)^2 &  & N_{dec}<n<N
    \end{aligned}\right.
\]
\[
    \vec x(n+1)- \vec x(n) = \left\{\begin{aligned}
         & \dfrac{1}{2} a_{acc} \vec k t_s^2(2n+1)                      &  & 0<n<N_{acc}       \\
         & v\vec kt_s                                                   &  & N_{acc}<n<N_{dec} \\
         & v \vec k t_s-\frac{1}{2} a_{dec} \vec k t_s^2(2n+1-2N_{dec}) &  & N_{dec}<n<N
    \end{aligned}\right.
\]
% \marginnote{
%     计算三阶段所需时间
%     \begin{itemize}
%         \item 加速所需时间
%         \item 减速所需时间
%         \item 匀速所需时间
%     \end{itemize}
% }
% \[
%     % \vec x := \vec x_1- \vec x_0;\quad t:=t_{acc}+t_{v}+t_{dec}
% \]
% \[
%     \text{单位向量}:\vec k = \frac{\vec x}{\vert x\vert}
% \]
% \[
%     \text{采样点数}:N=\frac{t}{t_s}
% \]
% \[
%     \begin{aligned}
%          & \text{加速阶段结束点的序号}: N_{acc} = N\frac{t_{acc}}{t}     \\
%          & \text{减速阶段开始点的序号}: N_{dec} = N(1-\frac{t_{dec}}{t})
%     \end{aligned}
% \]
% \[
%     \vec x(n) = \left\{\begin{aligned}
%          & \dfrac{1}{2}\vec k a_{acc}(nt_s)^2                                                                                     &  & 0<n<N_{acc}       \\
%          & \dfrac{1}{2}\vec k a_{acc}(N_{acc}t_s)^2 + \vec k v(nt_s-t_{acc})                                                      &  & N_{acc}<n<N_{dec} \\
%          & \dfrac{1}{2}\vec k a_{acc}(N_{acc}t_s)^2 + \vec k v(N_{dec}t_s-t_{acc}) - \frac{1}{2}\vec k a_{dec}(nt_s-N_{dec}t_s)^2 &  & N_{dec}<n<N
%     \end{aligned}\right.
% \]
% 转为递推关系
% \[
%     \vec x(n+1)- \vec x(n) = \left\{\begin{aligned}
%          & \dfrac{1}{2} a_{acc} \vec k t_s^2(2n+1)                      &  & 0<n<N_{acc}       \\
%          & v\vec kt_s                                                   &  & N_{acc}<n<N_{dec} \\
%          & v \vec k t_s-\frac{1}{2} a_{dec} \vec k t_s^2(2n+1-2N_{dec}) &  & N_{dec}<n<N
%     \end{aligned}\right.
% \]
\end{document}